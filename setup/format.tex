% !Mode:: "TeX:UTF-8" 

\theoremstyle{plain}
\theorembodyfont{\song\rmfamily}
\theoremheaderfont{\hei\rmfamily}
\newtheorem{definition}{\hei 定义}[chapter]
\newtheorem{example}{\hei 例}[chapter]
\newtheorem{algo}{\hei 算法}[chapter]
\newtheorem{theorem}{\hei 定理}[chapter]
\newtheorem{axiom}{\hei 公理}[chapter]
\newtheorem{proposition}{\hei 命题}[chapter]
\newtheorem{lemma}{\hei 引理}[chapter]
\newtheorem{corollary}{\hei 推论}[chapter]
\newtheorem{remark}{\hei 注解}[chapter]
\newenvironment{proof}{\noindent{\hei 证明:}}{\hfill $ \square $ \vskip 4mm}
\theoremsymbol{$\square$}
\setlength{\theorempreskipamount}{0pt}
\setlength{\theorempostskipamount}{-2pt}

\allowdisplaybreaks[4]

%\CJKcaption {gb_452} 
%\CJKtilde
\setlength{\parindent}{2em}

\arraycolsep=1.6pt

\renewcommand\contentsname{\hei 目~~~~录}



%format
\CTEXsetup[number={\arabic{chapter}}]{chapter}
\renewcommand\chaptername{\thechapter.~}%按哈医大要求()
%控制章节名称的前缀/后缀等

\CTEXsetup[name={,.}]{chapter}
\setcounter{secnumdepth}{4} \setcounter{tocdepth}{2}
%========包含正文标题调整======
\titleformat{\chapter}{\center\hei\bfseries\xiaosan}{\chaptertitlename}{0.5em}{}%章节名前间距
\titlespacing{\chapter}{0pt}{-5.5mm}{8mm}
\titleformat{\section}{\song\bfseries\xiaosi}{\thesection}{0.5em}{}
\titlespacing{\section}{2em}{4.5mm}{4.5mm}
\titleformat{\subsection}{\song\bfseries\xiaosi}{\thesubsection}{0.5em}{}
\titlespacing{\subsection}{2em}{4.5mm}{4.5mm}
\titleformat{\subsubsection}{\song\bfseries\xiaosi}{\thesubsubsection}{0.0em}{}
\titlespacing{\subsubsection}{2em}{0mm}{0mm}
%哈医大要求二级/三级标题为 小四 宋 加黑 缩进2汉字=2em,此处em为全角。章节上下4mm间距——标准未提
%一级标题小三 黑 加粗
%加粗 \bfseries
\titlecontents{chapter}[3.8em]{\hspace{-3.8em}\song}{\thecontentslabel~~}{}{\titlerule*[4pt]{.}\contentspage}
\dottedcontents{section}[32pt]{}{21pt}{0.3pc}
\dottedcontents{subsection}[41pt]{}{30pt}{0.3pc}
% 按医大标准, 1级目录无缩进,以下各级标题均为2汉字

% 按工大标准, 缩小目录中各级标题之间的缩进,使它们相隔一个字符距离,也就是12pt
\makeatletter
\renewcommand*\l@chapter{\@dottedtocline{0}{0em}{5em}}%控制英文目录: 细点\@dottedtocline  粗点\@dottedtoclinebold
\renewcommand*\l@section{\@dottedtocline{1}{1em}{1.8em}}
\renewcommand*\l@subsection{\@dottedtocline{2}{2em}{2.5em}}


% 定义页眉和页脚
\newcommand{\makeheadrule}{	\rule[20pt]{\textwidth}{0.75pt} \\[5pt]
%		\rule{\textwidth}{0pt}
	}
\renewcommand{\headrule}{
	{%\if@fancyplain\let\headrulewidth\plainheadrulewidth\fi
		\makeheadrule}}
\pagestyle{fancyplain}

%去掉章节标题中的数字
%%不要注销这一行,否则页眉会变成:“第1章1  绪论”样式
%\renewcommand{\chaptermark}[1]{\markboth{\chaptertitlename~\ #1}{}}
\fancyhf{}
\renewcommand{\chaptermark}[1]{\markboth{#1}{}}%定义没有标题、号的页眉标题
%在book文件类别下,\leftmark自动存录各章之章名,\rightmark记录节标题

%% 页眉字号 工大要求 小五
%根据单双面打印设置不同的页眉;

\ifxueweidoctor
\fancyhead[CO]{\song \xiaowu\leftmark}
\fancyhead[CE]{\song \xiaowu 哈尔滨医科大学外科学(骨外)学位论文}%
\fancyfoot[C,C]{\xiaowu ~\thepage~}
\else
\fancyhead[CO]{\song \xiaowu \leftmark}
\fancyhead[CE]{\song \xiaowu 哈尔滨医科大学外科学(骨外)学位论文}%
\fancyfoot[C,C]{\xiaowu ~\thepage~}
\fi
% ==================

\renewcommand\frontmatter{\cleardoublepage
  \@mainmatterfalse
  \pagenumbering{Roman}}

% 调整罗列环境的布局
\setitemize{leftmargin=0em,itemsep=0em,partopsep=0em,parsep=0em,topsep=0em,itemindent=3em}
\setenumerate{leftmargin=0em,itemsep=0em,partopsep=0em,parsep=0em,topsep=0em,itemindent=3.5em}

\newcommand{\citeup}[1]{\textsuperscript{\cite{#1}}}

% 定制浮动图形和表格标题样式
\captionnamefont{\wuhao}
\captiontitlefont{\wuhao}
\captiondelim{~~}
%\captionstyle{\hang}
\hangcaption
\renewcommand{\subcapsize}{\wuhao}
\setlength{\abovecaptionskip}{0pt}
\setlength{\belowcaptionskip}{0pt}

% 自定义项目列表标签及格式 \begin{publist} 列表项 \end{publist}
\newcounter{pubctr} %自定义新计数器
\newenvironment{publist}{%%%%%定义新环境
\begin{list}{[\arabic{pubctr}]} %%标签格式
    {
     \usecounter{pubctr}
     \setlength{\leftmargin}{1.7em}     % 左边界 \leftmargin =\itemindent + \labelwidth + \labelsep
     \setlength{\itemindent}{0em}     % 标号缩进量
     \setlength{\labelsep}{0.5em}       % 标号和列表项之间的距离,默认0.5em
     \setlength{\rightmargin}{0em}    % 右边界
     \setlength{\topsep}{0ex}         % 列表到上下文的垂直距离
     \setlength{\parsep}{0ex}         % 段落间距
     \setlength{\itemsep}{0ex}        % 标签间距
     \setlength{\listparindent}{0pt} % 段落缩进量
    }}
{\end{list}}%%%%%

% 默认字体
\renewcommand\normalsize{
  \@setfontsize\normalsize{12pt}{12pt}
  \setlength\abovedisplayskip{4pt}
  \setlength\abovedisplayshortskip{4pt}
  \setlength\belowdisplayskip{\abovedisplayskip}
  \setlength\belowdisplayshortskip{\abovedisplayshortskip}
  \let\@listi\@listI}
  
% 设置行距和段落间垂直距离
\def\defaultfont{\renewcommand{\baselinestretch}{1.62}\normalsize\selectfont}
\renewcommand{\CJKglue}{\hskip 0.56pt plus 0.08\baselineskip} 
%加大字间距,使每行34个字,若要使得每行33个字,则将0.56pt替换为0.96pt。
\predisplaypenalty=0  %公式之前可以换页,公式出现在页面顶部

% 封面、摘要、版权、致谢格式定义
\def\ctitle#1{\def\@ctitle{#1}}\def\@ctitle{}
\def\cdegree#1{\def\@cdegree{#1}}\def\@cdegree{}
\def\caffil#1{\def\@caffil{#1}}\def\@caffil{}
\def\csubject#1{\def\@csubject{#1}}\def\@csubject{}
\def\cauthor#1{\def\@cauthor{#1}}\def\@cauthor{}
\def\csupervisor#1{\def\@csupervisor{#1}}\def\@csupervisor{}
\def\cassosupervisor#1{\def\@cassosupervisor{{\hei 副 \hfill 导 \hfill 师} & #1\\}}\def\@cassosupervisor{}
\def\ccosupervisor#1{\def\@ccosupervisor{{\hei 联 \hfill 合\hfill 导 \hfill 师} & #1\\}}\def\@ccosupervisor{}
\def\cdate#1{\def\@cdate{#1}}\def\@cdate{}
\long\def\cabstract#1{\long\def\@cabstract{#1}}\long\def\@cabstract{}
\def\ckeywords#1{\def\@ckeywords{#1}}\def\@ckeywords{}

\def\etitle#1{\def\@etitle{#1}}\def\@etitle{}
\def\edegree#1{\def\@edegree{#1}}\def\@edegree{}
\def\eaffil#1{\def\@eaffil{#1}}\def\@eaffil{}
\def\esubject#1{\def\@esubject{#1}}\def\@esubject{}
\def\eauthor#1{\def\@eauthor{#1}}\def\@eauthor{}
\def\esupervisor#1{\def\@esupervisor{#1}}\def\@esupervisor{}
\def\eassosupervisor#1{\def\@eassosupervisor{\textbf{Associate Supervisor:} & #1\\}}\def\@eassosupervisor{}
\def\ecosupervisor#1{\def\@ecosupervisor{\textbf{Co Supervisor:} & #1\\}}\def\@ecosupervisor{}
\def\edate#1{\def\@edate{#1}}\def\@edate{}
\long\def\eabstract#1{\long\def\@eabstract{#1}}\long\def\@eabstract{}
\long\def\NotationList#1{\long\def\@NotationList{#1}}\long\def\@NotationList{}
\def\ekeywords#1{\def\@ekeywords{#1}}\def\@ekeywords{}
\def\natclassifiedindex#1{\def\@natclassifiedindex{#1}}\def\@natclassifiedindex{}
\def\internatclassifiedindex#1{\def\@internatclassifiedindex{#1}}\def\@internatclassifiedindex{}
\def\statesecrets#1{\def\@statesecrets{#1}}\def\@statesecrets{}

% 定义封面



%==========================空白基线填充=============
% Usage: \medthss@int@fillinblank{(number of lines)}{(line width)}{(contents)}
%medthss@int@fillinblank{{1} {5 em} {啦啦啦}}
\def\medthss@tmp@len{0.56\textwidth}
\def\medthss@int@fillinblank#1#2#3{%
	\makebox[0pt][l]{\parbox[t]{#2}{\centering{#3}}}\mbox{}%
	\parbox[t]{#2}{%
		\newcount\medthss@tmp@linecount
		\medthss@tmp@linecount=#1
		\loop\ifnum\medthss@tmp@linecount>0
		% Fill specified space with underline on the bottom line. `\underline'
		% draws line on the baseline (not the bottom line), and this is why
		% `\uline' is used here instead.
		\ifnum\medthss@tmp@linecount=1
		\uline{\makebox[#2]{}}
		\else
		\uline{\makebox[#2]{}}\\
		\fi
		\advance\medthss@tmp@linecount by -1\relax
		\repeat%
	}%
}

% 设置封面主题 (cover).
%\renewcommand{\maketitle}{%
%	\medthss@int@pdfmark{\titlepagename}{titlepage}
%	\begin{titlepage}
%		% It will be nicer to use this line skip level in the title page.
%		\linespread{1.6}\selectfont
%		% Make the title page centered.
%		\begin{center}
%			% Emblem and inscription of the university, and type of thesis.
%			{%
%				\zihao{1}%
%				\includegraphics[height = 2.4em]{pkulogo}\hspace{0.4em}%
%				\raisebox{0.4em}{\includegraphics[height = 1.6em]{pkuword}}\\[0.8em]
%				{\bfseries{\cthesisname}}%
%			}
%			\vfill
%			% Title of the thesis.
%			{%
%				\zihao{2}{\label@ctitle}%
%				\medthss@int@fillinblank{2}{0.64\textwidth}{\textbf{\@ctitle}}%
%			}
%			\vfill
%			% Information about the author.
%			{%
%				% Slightly adjust the line skip when using new font size.
%				\sanhao\linespread{1.75}\selectfont
%				\def\medthss@tmp@len{0.56\textwidth}
%				\begin{tabular}{l@{\extracolsep{0.2em}}c}
%					{\bfseries\label@cauthor}		&
%					\medthss@int@fillinblank{1}{\medthss@tmp@len}{\song\@cauthor}		\\
%					{\bfseries\label@studentid}	&
%					\medthss@int@fillinblank{1}{\medthss@tmp@len}{\song\@studentid}	\\
%					{\bfseries\label@school}		&
%					\medthss@int@fillinblank{1}{\medthss@tmp@len}{\song\@school}		\\
%					{\bfseries\label@cmajor}		&
%					\medthss@int@fillinblank{1}{\medthss@tmp@len}{\song\@cmajor}		\\
%					{\bfseries\label@direction}	&
%					\medthss@int@fillinblank{1}{\medthss@tmp@len}{\song\@direction}	\\
%					{\bfseries\label@cmentor}		&
%					\medthss@int@fillinblank{1}{\medthss@tmp@len}{\song\@cmentor}		\\
%				\end{tabular}%
%			}
%			\vfill
%			% Date.
%=======================================







\def\makecover{
    \begin{titlepage}
    % 封面一
   \vspace*{0.8cm}
   \begin{center}
    \centerline{\xiaoyi\hei{硕士学位论文}}

    \vspace{1cm}

    \parbox[t][2.8cm][t]{\textwidth}{
    \begin{center}\erhao\hei\@ctitle\end{center} }

    \parbox[t][5.1cm][t]{\textwidth}{ %英文标题太长时可以采用\xiaoer
    \begin{center}\erhao{\@etitle}\end{center} }

    \parbox[t][7.4cm][t]{\textwidth}{
    \begin{center}\xiaoer\song{\@cauthor}\end{center}}

    \parbox[t][1.4cm][t]{\textwidth}{
    \begin{center}\song\sanhao{哈尔滨医科大学}\end{center} }
    
    {\song\sanhao{\@cdate}}

    \end{center}

    % 封二 空白页
    \ifxueweidoctor
      \newpage
      ~~~\vspace{1em}
      \thispagestyle{empty}
    \fi

    %内封
    \newpage
    \thispagestyle{empty}
    \pdfbookmark[0]{\@ctitle}{ctitlepage}
	\begin{center}
			{\song \xiaosi\linespread{1.6}\selectfont
			\begin{tabular}{@{}r@{:}l@{}}
			单位代码 & 10226\\
 			分~~类~~号 & \@internatclassifiedindex
			\end{tabular}}\hfill
			{\song \xiaosi
			\begin{tabular}{@{}r@{}l@{}}
			学号:~ & \@natclassifiedindex\\
 			\hspace{2em}&   ~~
			\end{tabular}}
   \parbox[t][2 cm][t]{\textwidth}{\rule{0pt}{16pt}\begin{center}
   		\includegraphics[width=11.1 cm]{SName.png} 
	   	\end{center} }
   	\\
   	\rule{0pt}{13pt}\\
    \parbox[t][2cm][t]{0.7\textwidth}{\begin{center}
    		\bfseries\song\fontsize{35pt}{35pt}硕\hfill 士\hfill 学\hfill 位\hfill 论\hfill 文\end{center}}\\
	\includegraphics[width=5.52 cm]{SLogo.png}
 %   \parbox[t][5cm][t]{\textwidth}{\erhao
 %   \begin{center} {\hei  \@ctitle}\end{center} }
	\parbox[t][9.8cm][b]{\textwidth}
     {\sihao
    \begin{center} \renewcommand{\arraystretch}{1.8} \kaishu
    \begin{tabular}{p{3cm} p{12cm}<{\centering}} %需要arry包
    {\hei\xiaosan 题\hfill 目}           & \medthss@int@fillinblank{1}{11.5cm}{\@ctitle}\\%若标题超行 手动替换\@ctitle
    %{\hei\xiaosan 专\hfill 业}           & \@csupervisor\\
	%\@cassosupervisor
	%\@ccosupervisor
	{~~} & \medthss@int@fillinblank{1}{11.5cm}{若超行手动修改这两格,不超用~占位}\\
    {\hei\xiaosan 学\hfill 科\hfill 专\hfill 业} & \medthss@int@fillinblank{1}{11.5cm}{\@cdegree}\\
    {\hei\xiaosan 学\hfill 位\hfill 类\hfill 别}           & \medthss@int@fillinblank{1}{11.5cm}{\@csubject}\\
    {\hei\xiaosan 硕\hfill 士\hfill 研\hfill 究\hfill 生} & \medthss@int@fillinblank{1}{11.5cm}{\@caffil}\\
    {\hei\xiaosan 指\hfill 导\hfill 教\hfill 师} & \medthss@int@fillinblank{1}{11.5cm}{\@csupervisor}\\
  %  {\hei 授予学位单位}                     & 哈尔滨医科大学
  %\medthss@int@fillinblank{1}{\medthss@tmp@len}{\@cauthor}
    \end{tabular} 
	\\
	\rule{0pt}{23pt}\\
	\sanhao\song\@cdate
	\renewcommand{\arraystretch}{1}
    \end{center} }
\end{center}

%%%%%%增加一空白页

    \newpage
    ~~~\vspace{1em}
    \thispagestyle{empty}


    % 英文封面
    \newpage
    \thispagestyle{empty}
    \pdfbookmark[0]{\uppercase{\@etitle}}{etitlepage}

    {
    \xiaosi\noindent Classified Index: \@natclassifiedindex \\
                  U.D.C:  \@internatclassifiedindex }
    \begin{center}
    \parbox[t][1.6cm][t]{\textwidth}{\begin{center} \end{center} }
    \parbox[t][3.5cm][t]{\textwidth}{\xiaoer
    \begin{center} {  Dissertation for the {\exueweier} Degree in \exueke}\end{center} } %与中文保持一致,删除in {\exueke}

    \parbox[t][7cm][t]{\textwidth}{\erhao
    \begin{center} { \bfseries \@etitle}\end{center} }

%★★★★若信息内容不太长,不会引起信息内容分行时,使用tabular环境,否则使用下面的tabularx环境。
    {\sihao\renewcommand{\arraystretch}{1.3}
    \begin{tabular}{@{}l@{~}l@{}}
    \textbf{Candidate:}                     &  \@eauthor\\
    \textbf{Supervisor:}                    &  \@esupervisor\\
	  \@eassosupervisor
	  \@ecosupervisor
    \textbf{Academic Degree Applied for:}   &  \@edegree\\
    \textbf{Specialty:}                     &  \@esubject\\
    \textbf{Affiliation:}                   &  \@eaffil\\
    \textbf{Date of Defence:}               &  \@edate\\
    \textbf{Degree-Conferring-Institution:} &  Harbin Institute of Technology
    \end{tabular}\renewcommand{\arraystretch}{1}}

    %{\sihao\renewcommand{\arraystretch}{1.3}
    %\begin{tabularx}{\textwidth}{@{}l@{~}X@{}}
    %\textbf{Candidate:}                     &  \@eauthor\\
    %\textbf{Supervisor:}                    &  \@esupervisor\\
		%\@eassosupervisor
	  %\@ecosupervisor
    %\textbf{Academic Degree Applied for:}   &  \@edegree\\
    %\textbf{Specialty:}                     &  \@esubject\\
    %\textbf{Affiliation:}                   &  \@eaffil\\
    %\textbf{Date of Defence:}               &  \@edate\\
    %\textbf{Degree-Conferring-Institution:} &  Harbin Institute of Technology
    %\end{tabularx}\renewcommand{\arraystretch}{1}}

    \end{center}
    \end{titlepage}

%%%%%%增加一空白页
  \ifxueweidoctor
    \newpage
    ~~~\vspace{1em}
    \thispagestyle{empty}
  \fi
%%%%%%%%%%%%%%%%%%%   Abstract and keywords  %%%%%此处废弃不用%%%%
\clearpage
%\BiAppendixChapter{哈}{shegming}
%\thispagestyle{empty}
%本人郑重声明: 所呈交的学位论文,是本人在导师的指导下进行研究工作所取得的成果,实验数据与结果真实可靠。除文中已经注明引用的内容外,本论文不含任何其他个人或集体已经发表或撰写过的研究成果。对本文研究做出重要贡献的个人和集体,均已在文中以明确方式标明。本声明的法律结果由本人承担。\\

%论文作者签名:

%\hspace{6cm}	 日期:\hspace{5em} 年\hspace{2em}月\hspace{2em}日\hspace{2em}

%\setcounter{page}{1}
%\song\defaultfont
%\@cshengming
%\vspace{\baselineskip}
%增加空白页
%    \newpage
%~~~\vspace{1em}
%\thispagestyle{empty}
%\hangafter=1\hangindent=51pt\noindent
%{\hei 关键词}:\@ckeywords
%\clearpage
%\BiAppendixChapter{声明2}{shegming2}
%\thispagestyle{empty}

%\clearpage
%\BiAppendixChapter{缩略词表}{Abs. Table}
%%%%%%%%%%%%%%%%%%%   Abstract and keywords  %%%%%%%%%%%%%%%%%%%%%%%
%\clearpage

%\BiAppendixChapter{摘\quad 要}{Abstract (In Chinese)}
%============页面jish
%\setcounter{page}{1}
%\song\defaultfont
%\@cabstract
%\vspace{\baselineskip}

%\hangafter=1\hangindent=51pt\noindent
%{\hei 关键词}:\@ckeywords

%%%%%%%%%%%%%%%%%%%   English Abstract  %%%%%%%%%%%%%%%%%%%%%%%%%%%%%%
\clearpage
%目录特殊调整%%%%%%%%%%%%%%%%%%%%%%%%%%%%%
%\phantomsection
%\markboth{Abstract}{Abstract}
%\addcontentsline{toc}{chapter}{\xiaosi ABSTRACT}
%\addcontentsline{toe}{chapter}{\bfseries \xiaosi Abstract (In English)}  
\addtocontents{toc}{\vspace{\baselineskip}}
\addtocontents{toe}{\vspace{\baselineskip}}
%\chapter*{\bf Abstract}
%\@eabstract
%\vspace{\baselineskip}

%\hangafter=1\hangindent=60pt\noindent{\textbf{Keywords:}} 
}%此行不要注释掉%%%%%%%%%%%%%%%%%%%%%%%%%%%%%%%%%%%%%%%%%%%%%%%%%%%%%%%%%%
%\@ekeywords
\clearpage

%%%%%%%%%%%%%%%%%%%%%%%%%%%%%%%%%%%%%%%%%%%%%%%%%%%%%%%%%%%%%%%
% 英文目录格式
\def\@dotsep{0.75}           % 定义英文目录的点间距
\setlength\leftmargini {0pt}
\setlength\leftmarginii {0pt}
\setlength\leftmarginiii {0pt}
\setlength\leftmarginiv {0pt}
\setlength\leftmarginv {0pt}
\setlength\leftmarginvi {0pt}

\def\engcontentsname{\bfseries Contents}
\newcommand\tableofengcontents{
   \pdfbookmark[0]{Contents}{econtent}
     \@restonecolfalse
   \chapter*{\engcontentsname  %chapter*上移一行,避免在toc中出现。
       \@mkboth{%
          \engcontentsname}{\engcontentsname}}
   \@starttoc{toe}%
   \if@restonecol\twocolumn\fi
   }

\urlstyle{same}  %论文中引用的网址的字体默认与正文中字体不一致,这里修正为一致的。

\renewcommand\endtable{\vspace{-4mm}\end@float}

\makeatother
